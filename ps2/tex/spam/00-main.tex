\item \points{20} {\bf Spam classification}

In this problem, we will use the naive Bayes algorithm to
build a spam classifier.

In recent years, spam on electronic media has been a growing concern.  Here, we'll build a classifier to distinguish
between real messages, and spam messages. For this class, we will be building a classifier to detect SMS spam messages. We will be using an SMS spam dataset developed by Tiago A. Almedia and José María Gómez Hidalgo which is publicly available on \url{http://www.dt.fee.unicamp.br/~tiago/smsspamcollection} \footnote{Almeida, T.A., Gómez Hidalgo, J.M., Yamakami, A. Contributions to the Study of SMS Spam Filtering: New Collection and Results.  Proceedings of the 2011 ACM Symposium on Document Engineering (DOCENG'11), Mountain View, CA, USA, 2011.}

We have split this dataset into training and testing sets and have included them in this assignment as \texttt{src/spam/spam\_train.tsv} and \texttt{src/spam/spam\_test.tsv}. See \texttt{src/spam/spam\_readme.txt} for more details about this dataset. Please refrain from redistributing these dataset files. The goal of this assignment is to build a classifier from scratch that can tell the difference the spam and non-spam messages using the text of the SMS message.

Note that your code should be run inside the \texttt{src/spam/} directory.

\begin{enumerate}
  \item \subquestionpoints{5}
Implement code for processing the the spam messages into numpy arrays that can be fed into machine learning models. Do this by completing the \texttt{get\_words}, \texttt{create\_dictionary}, and \texttt{transform\_text} functions within our provided \texttt{src/spam/spam.py}. Do note the corresponding comments for each function for instructions on what specific processing is required.

The provided code will then run your functions and save the resulting dictionary into \texttt{spam\_dictionary} and a sample of the resulting training matrix into\\
\texttt{spam\_sample\_train\_matrix}.
In your writeup, report the vocabulary size after the pre-processing step. You do not need to include any other output for this subquestion.



  \ifnum\solutions=1 {
    \begin{answer}
Size of dictionary:  1722
\end{answer}

  } \fi

  \item \subquestionpoints{10}
In this question you are going to implement a naive Bayes classifier for spam
classification with {\bf multinomial event model} and Laplace smoothing.

Code your implementation by completing the \texttt{fit\_naive\_bayes\_model}
and \\\texttt{predict\_from\_naive\_bayes\_model} functions in
\texttt{src/spam/spam.py}.

Now \texttt{src/spam/spam.py} should be able to train a Naive Bayes model,
compute your prediction accuracy and then save your resulting predictions
to \texttt{spam\_naive\_bayes\_predictions}.

In your writeup, report the accuracy of the trained model on the \textbf{test set}.

{\bf Remark.} If you implement naive Bayes the straightforward way, you will find
that the computed $p(x|y) = \prod_i p(x_i | y)$ often equals zero.  This is
because $p(x|y)$, which is the product of many numbers less than one, is a very
small  number. The standard computer representation of real numbers cannot
handle numbers that are too small, and instead rounds them off to zero.  (This
is called  ``underflow.'')  You'll have to find a way to compute Naive Bayes'
predicted  class labels without explicitly representing very small numbers such
as $p(x|y)$.

[\textbf{Hint:} Think about using logarithms.]


  \ifnum\solutions=1 {
    \begin{answer}

\end{answer}

  } \fi

  \item \subquestionpoints{5}
Intuitively, some tokens may be particularly indicative of an SMS being
in a particular class.  We can try to get an informal sense of how indicative
token $i$ is for the SPAM class by looking at:
\begin{equation*}
  \log \frac{p(x_j = i \mid y=1)}{p(x_j = i \mid y=0)}
  = \log\left(\frac{P(\hbox{token $i$} \mid \hbox{email is SPAM})}
    {P(\hbox{token $i$} \mid \hbox{email is NOTSPAM})}\right).
\end{equation*}

Complete the \texttt{get\_top\_five\_naive\_bayes\_words} function within the provided code using the above formula in order to obtain the 5 most indicative tokens.
Report the top five words in your writeup.


  \ifnum\solutions=1 {
    \begin{answer}
['claim', 'won', 'prize', 'tone', 'urgent!']
\end{answer}

  } \fi
\end{enumerate}
