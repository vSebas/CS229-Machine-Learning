\item \points{25} {\bf Linear Classifiers (GDA)}

In PSET 1, you covered logistic regression in problem 3. In this problem, we apply a generative linear classifier, Gaussian discriminant
analysis (GDA), on the same datasets. Both of the algorithms find a linear decision boundary that
separates the data into two classes, but make different assumptions. Our goal
in this problem is to get a deeper understanding of the similarities and
differences (and, strengths and weaknesses) of these two algorithms.

For this problem, we will consider the same two datasets, along with starter codes provided in the following
files:
\begin{center}
\begin{itemize} %[label=\roman*.]
	\item \url{src/linearclass/ds1_{train,valid}.csv}
	\item \url{src/linearclass/ds2_{train,valid}.csv}
        \item \url{src/linearclass/gda.py}
\end{itemize}
\end{center}
Recall that each file contains $\nexp$ examples, one example $(x^{(i)}, y^{(i)})$ per row.
In particular, the $i$-th row contains columns $x^{(i)}_1\in\Re$,
$x^{(i)}_2\in\Re$, and $y^{(i)}\in\{0, 1\}$.

\begin{enumerate}

	\input{linearclass/01-gda}
        \ifnum\solutions=1 {
            \begin{answer}

$$
p(y=1\mid x;\phi,\mu_0,\mu_1,\Sigma)
= \frac{p(x\mid y=1)p(y=1)}
       {p(x\mid y=1)p(y=1) + p(x\mid y=0)p(y=0)}
$$
Rewrite the posterior following $\frac{a}{a+b} = \frac{1}{a^{-1}} \frac{1}{a+b} = \frac{1}{1+a^{-1}b}$
$$
p(y=1\mid x;\phi,\mu_0,\mu_1,\Sigma)
= \frac{1}
       {1 + \frac{p(x\mid y=0)p(y=0)}{p(x\mid y=1)p(y=1)}}
$$
Then, to comply with the logistics form:
$$
p(y=1\mid x;\phi,\mu_0,\mu_1,\Sigma)
= \frac{1}
       {1 + \exp \log \left( \frac{p(x\mid y=0)p(y=0)}{p(x\mid y=1)p(y=1)} \right)}
$$
$$
\log\frac{p(x\mid y=0)p(y=0)}{p(x\mid y=1)p(y=1)}
= \log \frac{p(x\mid y=0)}{p(x\mid y=1)}
+ \log \frac{p(y=0)}{p(y=1)}
$$
And
$$
p(x\mid y=i)
= \frac{1}{(2\pi)^{d/2}|\Sigma|^{1/2}}
  \exp\!\left(-\frac12(x-\mu_i)^{\top}\Sigma^{-1}(x-\mu_i)\right),
  \quad i\in\{0,1\}.
$$
So
$$
\begin{align}
\log \frac{p(x\mid y=0)}{p(x\mid y=1)}
&= -\frac{1}{2}\Big[
(x-\mu_0)^{\top}\Sigma^{-1}(x-\mu_0)
- (x-\mu_1)^{\top}\Sigma^{-1}(x-\mu_1)
\Big] \\
&= x^{\top}\Sigma^{-1}(\mu_0-\mu_1)
   + \tfrac{1}{2}(\mu_1^{\top}\Sigma^{-1}\mu_1 - \mu_0^{\top}\Sigma^{-1}\mu_0)
\end{align}
$$
$$
\log \frac{p(y=0)}{p(y=1)} = \log \frac{1-\phi}{\phi} = - \log \frac{\phi}{1-\phi}
$$
$$
\exp \log \left( \frac{p(x\mid y=0)p(y=0)}{p(x\mid y=1)p(y=1)} \right)
= \exp \left(
(\mu_0-\mu_1)^{\top}\Sigma^{-1}x
   + \tfrac{1}{2}(\mu_1^{\top}\Sigma^{-1}\mu_1 - \mu_0^{\top}\Sigma^{-1}\mu_0)
+ \log \frac{1-\phi}{\phi}
\right)
$$
Then, the values of $\theta$ and $\theta_{0}$ to match the logistic form are:

$$
\boxed{
\theta = \Sigma^{-1}(\mu_{1}-\mu_{0})
}
$$
$$
\boxed{
\theta_{0} 
= \tfrac{1}{2}(\mu_0^{\top}\Sigma^{-1}\mu_0 - \mu_1^{\top}\Sigma^{-1}\mu_1)
+ \log \frac{\phi}{1-\phi}
}
$$
\end{answer}

        }\fi

	\input{linearclass/02-gda-ll}
        \ifnum\solutions=1 {
            \begin{answer}

$$
\begin{aligned}
\ell(\theta)
&=\sum_{i=1}^n \log p(x^{(i)},y^{(i)};\theta)\\
&=\sum_{i=1}^n \log p(x^{(i)}\mid y^{(i)};\mu_0,\mu_1,\Sigma)+\sum_{i=1}^n \log p(y^{(i)};\phi)\\
&=\sum_{i=1}^n\Bigg[-\frac{d}{2}\log(2\pi)-\frac{1}{2}\log|\Sigma|
-\frac{1}{2}\big(x^{(i)}-\mu_{y^{(i)}}\big)^\top\Sigma^{-1}\big(x^{(i)}-\mu_{y^{(i)}}\big)\Bigg]\\
&\quad +\sum_{i=1}^n\Big[y^{(i)}\log\phi+\big(1-y^{(i)}\big)\log(1-\phi)\Big]
\end{aligned}
$$
Constants can be ignored when maximizing the log-likelihood, so we have
$$
\begin{aligned}
\ell(\theta)
&=-\frac{n}{2}\log|\Sigma| - \frac{1}{2}\sum_{i=1}^n\big(x^{(i)}-\mu_{
y^{(i)}}\big)^\top\Sigma^{-1}\big(x^{(i)}-\mu_{y^{(i)}}\big)\\
&\quad +\sum_{i=1}^n\Big[y^{(i)}\log\phi+\big(1-y^{(i)}\big)\log(1-\phi)\Big]
\end{aligned}
$$
For $\phi$, we have
$$
\begin{aligned}
\frac{\partial}{\partial \phi}\ell(\theta)
&=\sum_{i=1}^n\Big[\frac{y^{(i)}}{\phi}-\frac{1-y^{(i)}}{1-\phi}\Big]\\
&=\frac{1}{\phi(1-\phi)}\Big[\sum_{i=1}^n y^{(i)} - n\phi\Big]
\end{aligned}
$$
Setting this to zero,
$$
\boxed{
\phi = \frac{1}{n}\sum_{i=1}^n y^{(i)}
}
$$
For $\mu_0$, we care for those with $y^{(i)}=0$, so
$$
\ell(\mu_0) = -\frac{1}{2}\sum_{i=1}^n \mathbf{1}\{y^{(i)}=0\}
\big(x^{(i)}-\mu_0\big)^\top\Sigma^{-1}\big(x^{(i)}-\mu_0\big) + \text{const}
$$
$$
\begin{aligned}
\frac{\partial \ell}{\partial \mu_0}
&= -\frac{1}{2}\sum_{i=1}^n \mathbf{1}\{y^{(i)}=0\}\cdot 
\big(-2\Sigma^{-1}(x^{(i)}-\mu_0)\big)\\
&=\Sigma^{-1}\sum_{i=1}^n \mathbf{1}\{y^{(i)}=0\}\big(x^{(i)}-\mu_0\big)
\end{aligned}
$$
Setting this to zero,
$$
\sum_{i=1}^n \mathbf{1}\{y^{(i)}=0\}x^{(i)}=
\left(\sum_{i=1}^n \mathbf{1}\{y^{(i)}=0\}\right)\mu_0
\;\therefore\;
\boxed{\;\displaystyle \mu_0=\frac{\sum_{i=1}^n \mathbf{1}\{y^{(i)}=0\}x^{(i)}}{\sum_{i=1}^n \mathbf{1}\{y^{(i)}=0\}}\; }
$$
Similarly, for $\mu_1$, we have
$$
\boxed{\;\displaystyle \mu_1=\frac{\sum_{i=1}^n \mathbf{1}\{y^{(i)}=1\}x^{(i)}}{\sum_{i=1}^n \mathbf{1}\{y^{(i)}=1\}}\; }.
$$
For $\Sigma$, we have
$$
\ell(\Sigma)=-\frac{n}{2}\log|\Sigma|-\frac{1}{2}\sum_{i=1}^n
\big(x^{(i)}-\mu_{y^{(i)}}\big)^\top\Sigma^{-1}\big(x^{(i)}-\mu_{y^{(i)}}\big)+\text{const}
$$
The quadratic term can be rewritten as a trace, so
$$
\ell(\Sigma)=-\frac{n}{2}\log|\Sigma|-\frac{1}{2}\sum_{i=1}^n
\text{tr}(\Sigma^{-1}\big(x^{(i)}-\mu_{y^{(i)}}\big)\big(x^{(i)}-\mu_{y^{(i)}}\big)^\top)
$$
$$
\frac{\partial \ell}{\partial \Sigma}
=-\frac{n}{2}\Sigma^{-1}+\frac{1}{2}\sum_{i=1}^n\Sigma^{-1}\big(x^{(i)}-\mu_{y^{(i)}}\big)\big(x^{(i)}-\mu_{y^{(i)}}\big)^\top\Sigma^{-1}.
$$
Setting this to zero,
$$
\boxed{
\Sigma = \frac{1}{n}\sum_{i=1}^n \big(x^{(i)}-\mu_{y^{(i)}}\big)\big(x^{(i)}-\mu_{y^{(i)}}\big)^\top
}
$$
\end{answer}

        } \fi

	\item \subquestionpoints{10} \textbf{Coding problem.}
In \texttt{src/linearclass/gda.py}, fill in the code to
calculate $\phi$, $\mu_{0}$, $\mu_{1}$, and $\Sigma$, use these parameters
to derive $\theta$, and use the resulting GDA model to make predictions on the
validation set. Make sure to write your model's predictions on
the validation set to the file specified in the code.

Include two plots of the \textbf{validation data} for both datasets with $x_1$ on the horizontal axis and $x_2$ on the vertical axis.
To visualize the two classes, use a different symbol for examples $x^{(i)}$
with $y^{(i)} = 0$ than for those with $y^{(i)} = 1$. On the same figures, plot the decision boundary
found by GDA (i.e, line corresponding to $p(y|x) = 0.5$).

Note that your code should be run inside the \texttt{src/linearclass/} directory.


        \ifnum\solutions=1 {
            \begin{answer}
\ref{fig:GDA_pred_1} and \ref{fig:GDA_pred_2}
\[
\begin{figure}
    \centering
    \includegraphics[width=1\linewidth]{/home/saveasmtz/Documents/CS229/ps2/tex/imgs/gda_pred_1.png}
    \caption{True Counts vs Predicted Counts}
    \label{fig:GDA_pred_1}
\end{figure}

\begin{figure}
    \centering
    \includegraphics[width=1\linewidth]{/home/saveasmtz/Documents/CS229/ps2/tex/imgs/gda_pred_2.png}
    \caption{True Counts vs Predicted Counts}
    \label{fig:GDA_pred_2}
\end{figure}
\]

\end{answer}

        } \fi

	\item \subquestionpoints{2}
For both datasets, compare the validation set plots obtained in part (c) and PSET 1, Problem 3, part (b)
from GDA and logistic regression respectively, and briefly comment on your observation
in a couple of lines.
On which dataset does GDA seem to perform worse than logistic regression? Why might this be the case?


        \ifnum\solutions=1 {
            \begin{answer}
Dataset 1, maybe because the covariance matrices for each classifier are not really the same.
\end{answer}

        } \fi

	\item \subquestionpoints{1} For the dataset where GDA performed worse in
part (d), can you find a transformation of the $x^{(i)}$'s such
that GDA performs significantly better? What might this transformation be?


        \ifnum\solutions=1{
            \begin{answer}
Scale the $x_2$ dimension by taking its log, which reduces its spread and makes the classes more similar.
\end{answer}

        }\fi

\end{enumerate}
