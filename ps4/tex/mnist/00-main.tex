\item \points{30} {\bf Neural Networks: MNIST image classification}

In this problem, you will implement a simple neural network
to classify grayscale images of handwritten digits (0 - 9) from
the MNIST dataset. The dataset contains 60,000 training images and
10,000 testing images of handwritten digits, 0 - 9. Each image is
28$\times$28 pixels in size, and is generally represented as a flat
vector of 784 numbers. It also includes labels for each example, a number
indicating the actual digit (0 - 9) handwritten in that image. A sample of
a few such images are shown below.

\begin{center}
\includegraphics[scale=0.5]{mnist/mnist_plot}
\end{center}

The data and starter code for this problem can be found in

\begin{itemize}
\item \texttt{src/mnist/nn.py}
\item \texttt{src/mnist/images\_train.csv}
\item \texttt{src/mnist/labels\_train.csv}
\item \texttt{src/mnist/images\_test.csv}
\item \texttt{src/mnist/labels\_test.csv}
\end{itemize}

The dataset files are initially provided in compressed \texttt{.gz} format. Be sure to unzip them before use so that the files are accessible as \texttt{.csv} files. To unzip the files:
\begin{itemize}
\item On Mac/Linux: \texttt{gunzip images\_train.csv.gz} (repeat for each \texttt{.gz} file)
\item On Windows: Use a tool like 7-Zip, WinRAR, or the built-in extraction feature
\end{itemize}
The starter code splits the set
of 60,000 training images and labels into a set of 50,000 examples as
the training set, and 10,000 examples for dev set.

To start, you will implement a neural network with a single hidden layer
and cross entropy loss, and train it with the provided data set. You will use the
sigmoid function as activation for the hidden layer and use the cross-entropy loss for multi-class classification. For a single example $(x, y)$, the cross-entropy loss is given by:
%\tnote{change the notation here to make it more consistent with the lecture }
$$\ell_\textup{CE}(\bar{h}_{\theta}(x),y) = - \log\left(\frac{\exp(\bar{h}_{\theta}(x)_{y})}{\sum_{s=1}^{k}\exp({\bar{h}_{\theta}(x)}_s)}\right),$$
where $\bar{h}_{\theta}(x) \in \mathbb{R}^{k}$ denotes the logits, i.e., the output of the model on a training example $x$, and $\bar{h}_{\theta}(x)_y$ is the $y$-th coordinate of the vector $\bar{h}_{\theta}(x)$ (with $y \in \{1, \dots, k\}$ serving as an index).

%\tnote{let's call the one-hot vector $e_y$ so that we don't have to overload the notation. Can still keep the $\hat{y}$. The cross-entropy loss should be defined the same as in the lecture notes}


We have labeled data $(x^{(i)}, y^{(i)})_{i=1}^n$, where $x^{(i)} \in \mathbb{R}^d$, and $y^{(i)} \in \{1,\dots, k\}$ is the ground truth label. Note that, in this specific application, we pass in the \textit{flattened} images as input. In other words, if each original input image is in $\mathbb{R}^{r \times r}$, we have $x^{(i)} \in \mathbb{R}^{r^2}$.

Additionally, let $m$ be the number of hidden units in the neural network, so that weight matrices $W^{[1]} \in \mathbb{R}^{d \times m}$ and $W^{[2]} \in \mathbb{R}^{m \times k}$.\footnote{Please note that the dimension of the weight matrices is different from those in the lecture notes, but we also multiply ${W^{[1]}}^\top$ instead of $W^{[1]}$ in the matrix multiplication layer.  Such a change of notation is mostly for some consistence with the convention in the code.} We also have biases $b^{[1]} \in \mathbb{R}^m$ and $b^{[2]} \in \mathbb{R}^k$. The parameters of the model $\theta$ is $(W^{[1]},W^{[2]},b^{[1]},b^{[2]})$.

We define the following neural network with a single hidden layer, which we refer to as the \textbf{Linear Layer NN}. For a single input $x^{(i)}$, the forward propagation equations are:

\begin{align}
  a^{(i)} &= \sigma \left( {W^{[1]}}^\top x^{(i)}  + b^{[1]} \right)  \in \mathbb{R}^m \nonumber \\
  \bar{h}_{\theta}(x^{(i)})&= {W^{[2]}}^\top a^{(i)} + b^{[2]} \in \mathbb{R}^k \nonumber \\
  {h}_{\theta}(x^{(i)}) &=  \mathrm{softmax}(\bar{h}_{\theta}(x^{(i)})) \in \mathbb{R}^k \nonumber
\end{align}
where $\sigma$ is the sigmoid function. The softmax function maps $\mathbb{R}^k \to \mathbb{R}^k$ and is defined as:
$$\mathrm{softmax}(z)_i = \frac{\exp(z_i)}{\sum_{j=1}^{k}\exp(z_j)}$$
for each component $i \in \{1, \dots, k\}$. 

For $\nexp$ training examples, we average the cross entropy loss over the $\nexp$ examples.
  \begin{equation*}
  J(W^{[1]},W^{[2]},b^{[1]},b^{[2]}) = \frac{1}{\nexp}\sum_{i=1}^\nexp \ell_\textup{CE}(\bar{h}_{\theta}(x^{(i)}),y^{(i)})  = - \frac{1}{\nexp}\sum_{i=1}^\nexp \log\left(\frac{\exp(\bar{h}_{\theta}(x^{(i)})_{y^{(i)}})}{\sum_{s=1}^{k}\exp({\bar{h}_{\theta}(x^{(i)})}_s)}\right).
  \end{equation*}

Suppose $e_y\in \Re^k$ is the one-hot embedding/representation of the discrete label $y$, where the $y$-th entry is 1 and all other entries are zeros. We can also write the loss function in the following way:
  \begin{equation*}
  J(W^{[1]},W^{[2]},b^{[1]},b^{[2]}) = - \frac{1}{\nexp}\sum_{i=1}^\nexp e_{\ysi}^\top\log\left(h_\theta(x^{(i)})\right).
  \end{equation*}
Here $\log(\cdot)$ is applied entry-wise to the vector $h_\theta(\xsi)$. The starter code already converts labels into one-hot representations for you.


Instead of batch gradient descent or stochastic gradient descent, the common practice
is to use mini-batch gradient descent for deep learning tasks. Concretely, we randomly sample $B$ examples $(x^{(i_k)}, y^{(i_k)})_{k=1}^B$ from $(x^{(i)}, y^{(i)})_{i=1}^n$. In this case, the
mini-batch cost function with batch-size $B$ is defined as follows:

  \begin{equation*}
  J_{MB} = \frac{1}{B}\sum_{k=1}^B \ell_\textup{CE}(\bar{h}_{\theta}(x^{(i_k)}),y^{(i_k)})
  \end{equation*}
where $B$ is the batch size, i.e., the number of training examples in each mini-batch.

\begin{enumerate}
  \input{mnist/01-grad}

\ifnum\solutions=1 {
  \begin{answer}

We start with the definition of the cross-entropy loss:
$$
\begin{align*}
\ell_{\text{CE}}(t,y)
    &= -\log\left( \frac{\exp(t_{y})}{\sum_{s=1}^{k}\exp(t_{s})} \right) \\
    &= -\log(\exp(t_{y})) + \log\left( \sum_{s=1}^{k}\exp(t_{s}) \right) \\
    &= -t_{y} + \log\left( \sum_{s=1}^{k}\exp(t_{s}) \right)
\end{align*}
$$

Take derivatives w.r.t. the $i$-th coordinate of $t$:

$$
\begin{align*}
\frac{d}{dt_{i}}(-t_{y})
    &= -\mathbbm{1}\{i=y\}
    = -(e_{y})_{i} 
    & \text{(one-hot vector)} \\
\frac{d}{dt_{i}} \log\left( \sum_{s=1}^{k}\exp(t_{s}) \right)
    &= \frac{\exp(t_{i})}{\sum_{s=1}^{k}\exp(t_{s})} 
    & \text{(only $\exp(t_i)$ depends on $t_i$)} \\
\frac{d}{dt_{i}} \ell_{\text{CE}}(t,y)
    &= -(e_{y})_{i} + \frac{\exp(t_{i})}{\sum_{s=1}^{k}\exp(t_{s})} \\
    &= p_{i} - (e_{y})_{i}
\end{align*}
$$

For all coordinates:    
$$
\boxed{
\nabla_{t}\ell_{\text{CE}}(t,y) = p - e_{y}
}
$$


\end{answer}

} \fi

  \input{mnist/02-unregularized}

\ifnum\solutions=1 {
  \begin{answer}

\begin{center}
\includegraphics[scale=0.8]{mnist/baseline.pdf}
\end{center}

\end{answer}
   
  

} \fi

  \input{mnist/03-regularized}

\ifnum\solutions=1 {
  \begin{answer}

\begin{center}
\includegraphics[scale=0.8]{mnist/regularized.pdf}
\end{center}

\end{answer}
   
  

} \fi


  \item \points{3}
% All this while you should have stayed away from the test data completely. Now that
% you have convinced yourself that the model is working as expected (i.e., the
% observations you made in the previous part matches what you learnt in class
% about regularization), it is finally time to measure the model performance on
% the test set. Once we measure the test set performance, we report it (whatever
% value it may be), and NOT go back and refine the model any further.

% Initialize your model from the parameters saved in part (a) (i.e., the non-regularized
% model), and evaluate the model performance on the test data. Repeat this using the
% parameters saved in part (b) (i.e., the regularized model).

% Report your test accuracy for both regularized model and non-regularized model.  
% Briefly (in one sentence) explain why this outcome makes sense.
% You should have accuracy close to 0.92870 without regularization, and 0.96760 with regularization.
% Note: these accuracies assume you implement the code with the matrix dimensions as specified in
% the comments, which is not the same way as specified in your code. Even if you do not precisely these
% numbers, you should observe good accuracy and better test accuracy with regularization.

Now that you have convinced yourself that the model is working as expected 
(i.e., the observations you made in the previous part match what you learnt in class 
about regularization), it is time to report the test performance. 
Once the test performance is recorded, we report it as is and do not go back to refine the model.

The program in the previous parts already evaluates both models on the test set and prints their accuracies. 
Now, using the recorded test accuracies, compare the performance of the non-regularized and regularized models, 
and briefly (in one sentence) explain why the regularized model performs better on the test set.

You should have accuracy close to 0.92870 without regularization, and 0.96760 with regularization.
Note: these accuracies assume you implement the code with the matrix dimensions as specified in
the comments. Even if you do not match these
numbers exactly, you should observe good accuracy and better test accuracy with regularization.

  \ifnum\solutions=1 {
  \begin{answer}
    
    The regularized model performs better on the test set because the L2 shrinks weights and reduces
    overfitting to training data, so it generalizes better and achieves higher test accuracy.
\end{answer}
   
  

} \fi

%   \item \points{6} In this part, we compare the model complexity of the Linear Layer NN model with a convolution-based model. Let the original input image be a grayscale image with height and width $r$, meaning each image is of shape (1, $r$, $r$).

We define a \textbf{CNN} (convolutional neural network) version of our original model. We let the number of output classes be $k$, and we have the following operations:

1. \textit{Convolutional Layer}: We define \( a(i) \) as the output of applying a 2D convolutional layer with 32 filters (with bias) of size \( 3 \times 3 \), padding 1, stride 1, and ReLU activation on an input image \( x^{(i)} \in \mathbb{R}^{1 \times r \times r}\). % \in \mathbb{R}^{h \times h} 

\[a_1^{(i)} = \mathrm{ReLU}(\mathrm{Conv2D}(x^{(i)}))\]

2. \textit{Flattening}: After applying the convolution, we flatten \( a_1^{(i)} \) into a vector.

\[a_2^{(i)} = \mathrm{Flatten}(a_1^{(i)}))\]

3. \textit{Linear Transformation}: We apply a linear transformation to the flattened vector, mapping it to $\mathbb{R}^k$
\[
\bar{h}_{\theta}(x^{(i)}) = {W^{[2]}}^\top a_2^{(i)} + b^{[2]}
\]

4. \textit{Softmax Output}: We apply the softmax activation to obtain our final output.
\[
{h}_{\theta}(x^{(i)}) = \mathrm{softmax}(\bar{h}_{\theta}(x^{(i)})) 
\]

\begin{enumerate}
    \item \subquestionpoints{1} In the CNN model, what are the dimensions of $a_1^{(i)}$ right after the convolutional layer (before the flattening operation) in terms of $r$?

    \item \subquestionpoints{1} In the CNN model, how many parameters are there in terms of in terms of the input height/width $r$ and the number of output classes $k$? % How many are there in the MNIST example ($r=28, k=10$)?

    \item \subquestionpoints{1} In the Linear Layer NN model, how many parameters are there in terms of the input height/width $r$, the number of hidden units $m$, and the number of output classes $k$? % How many are there in the MNIST example ($r=28, m=300, k=10$)?

    \item \subquestionpoints{1} What is the maximum value of $m$ for which the Linear NN model has fewer parameters than the CNN model on the MNIST dataset ($r=28, k=10$)?
    
    \item \subquestionpoints{2} What is one advantage and one disadvantage of using the CNN compared to the Linear Layer NN?
    
\end{enumerate}
% \ifnum\solutions=1 {
%   \input{mnist/05-conv-sol}
% } \fi

 \end{enumerate}

