\item \points{20} {\bf Decision Trees and Gini Loss}

When growing a decision tree, we split the input space in a greedy, top-down, recursive manner.  Given a parent region $R_p$, we can choose a split $s_p(j, t)$ which yields two child regions $R_1 = \{X \mid x_j < t, X \in R_p\}$ and $R_2 = \{X \mid x_j \geq t, X\in R_p\}$.  Assuming we have defined a per region loss $L(R)$, at each branch we select the split that minimizes the weighted loss of the children:

\begin{align*}
    \min_{j,t} \frac{\lvert R_1 \rvert L(R_1) + \lvert R_2 \rvert L(R_2)}{\lvert R_1 \rvert + \lvert R_2 \rvert}
\end{align*}

When performing classification, a commonly used loss is the Gini loss, defined for the K-class classification problem as:

\begin{align*}
    G(R_m) = G(\vec{p}_m) = \sum_{k=1}^K p_{mk} (1 - p_{mk})
\end{align*}

Where $\vec{p}_m = \begin{bmatrix}p_{m1} & p_{m2} & \dots & p_{mK}\end{bmatrix}$ and $p_{mk}$ is the proportion of examples of class $k$ that are are present in region $R_m$.  However, we are oftentimes more interested in optimizing the final misclassification loss:

\begin{equation}
    \label{misclassificationloss}
    M(R_m) = M(\vec{p}_m) = 1 - \max_k p_{mk} 
\end{equation}

For the problems below, assume we are dealing with binary classification and that there are no degenerate cases where positive and negative datapoints overlap in the feature space.

\begin{enumerate}
    \item \subquestionpoints{5} Show that for any given split, the weighted Gini loss of the children can not exceed that of the parent. (\textbf{Hint}: first show that the Gini loss is strictly concave. And then use the fact that G is strictly concave meaning:
    \begin{align*}
        \forall p_1 \neq p_2, \forall t \in (0, 1): G(t p_1 + (1 - t) p_2) > t G(p_1) + (1 - t) G(p_2)
    \end{align*}
    
	\ifnum\solutions=1 {
	\begin{answer}
\subsubsection*{0-1 indicator function}
For each training example $(x_i,y_i)$, define margin $t_i := y_i f(x_i)$. Then $t_i>0$ when $x_i$ is correctly classified, and $t_i<0$ when it is misclassified.
Thus the misclassification indicator can be written as
\[
\mathbf{1}\{F(x_i)\neq y_i\}=\mathbf{1}\{t_i<0\}.
\]

\subsubsection*{Upper bound on 0-1 loss}
Consider two cases: 
\begin{itemize} 
    \item If \(t_i < 0\) (misclassification), then \(\mathbbm{1}[t_i < 0] = 1\) and \(\exp(-t_i) > 1\)
    \item If \(t_i \ge 0\) (correct), then \(\mathbbm{1}[t_i < 0] = 0\) and \(\exp(-t_i) \ge 0\)
\end{itemize} 
Therefore, in both cases, we have:
\[
\mathbbm{1}[t_i < 0] \le \exp(-t_i)
\]

\subsubsection*{Overall loss bound}
Averaging over the dataset,
\[
\frac{1}{n}\sum_{i=1}^n \mathbf{1}\{F(x_i)\neq y_i\}
\;\le\;
\frac{1}{n}\sum_{i=1}^n e^{-\,y_i f(x_i)}.
\]
    
\end{answer}
        } \fi
        
    \item \subquestionpoints{8} The weight for each data point $i$ at step $t+1$ can be defined recursively by
 		\begin{align*}
 			\alpha_{i, (t+1)}
 			= \frac{\alpha_{i, t} \exp(-\hat{w}_t f_t(x_i) y_i)}{Z_t},
 		\end{align*}
 	where $Z_t$ is a normalizing constant ensuring the weights sum to $1$
 	\begin{align*}
 		Z_t = \sum_{i=1}^n \alpha_{i, t} \exp(-\hat{w}_t f_t(x_i) y_i).
 	\end{align*}
 
 Show that
 \begin{align*}
 	\frac{1}{n} \sum_{i=1}^n \exp(-f(x_i)y_i)= \prod_{t=1}^T Z_t. 
 \end{align*}
    
	\ifnum\solutions=1 {
	\begin{answer}

The Gini loss remains unchanged after a split in two cases:

\textbf{Case 1}: One child node is empty (\(t = 0\) or \(t=1\)). Then:
\[
tG(\vec{p}_{1}) + (1-t)G(\vec{p}_{2}) =  0 \cdot G(\vec{p}_{1}) + G(\vec{p}_{2}) = G(\vec{p}),
\]

\textbf{Case 2}: Both child nodes have the same class proportions as the parent node (\(\vec{p}_{1} = \vec{p}_{2} = p\)):
\[
tG(\vec{p}_{1}) + (1-t)G(\vec{p}_{2}) =  tG(\vec{p}) + (1-t)G(\vec{p}) = G(\vec{p}),
\]

which is equal to the Gini loss before the split in both cases.

These cases do not contradict the strict concavity of the Gini loss, since concavity holds when the two child nodes differ in class proportions, and both cases above are the allowed equality conditions of concavity.

Furthermore, these cases do not prevent a fully grown tree from achivieng zero Gini loss. 
They may temporarilly stall the reduction at certain splits, but subsequent splits can still lead to pure nodes with zero Gini loss.
\end{answer}
        } \fi
        
    \item \subquestionpoints{9}  We showed above that training error is bounded above by $\prod_{t=1}^T Z_t$. At step $t$ the values $Z_1$, $Z_2$, $\ldots$, $Z_{t-1}$ are already fixed therefore at step $t$ we can choose $\alpha_t$ to minimize $Z_t$. Let
		\begin{align*}
			\varepsilon_t 
			= \sum_{i=1}^n \alpha_{i, t} 1_{\{f_t(x_i) \neq y_i\}}
		\end{align*}
	be the weighted training error for the weak classifier $f_t(x)$. Then we can re-write the formula for $Z_t$ as
	\begin{align*}
		Z_t = (1-\varepsilon_t) \exp(-\hat{w}_t) + \varepsilon_t \exp(\hat{w}_t).
	\end{align*}
 \begin{enumerate}
 	\item [(i)] [3 points] First find the value of  $\hat{w}_t$ that minimizes $Z_t$. Then show that the corresponding optimal value is
	\begin{align*}
		Z^{\text{opt}}_t
		= 2 \sqrt{\varepsilon_t (1-\varepsilon_t)}.
	\end{align*}
	\item [(ii)] [3 points] Assume we choose $Z_t$ as $Z^{\text{opt}}_t$ in part c (i). Then re-write $\varepsilon_t = 1/2 - \gamma_t$, where $\gamma_t > 0$ implies better than random and $\gamma_t < 0$ implies worse than random. Then show that
	\begin{align*}
		Z_t \le \exp(-2 \gamma_t^2).
	\end{align*}
	(You may want to use the fact that $\log(1 - x) \le  -x$ for $0 \le  x < 1$.)
	\item [(iii)] [3 points] Finally, show that if each classifier is better than random, i.e.,  $\gamma_t > \gamma$ for all $t$ and $\gamma > 0$, then
	\begin{align*}
			\varepsilon_{\text{training}}
		\le \exp(-2T \gamma^2),
	\end{align*}
	which shows that the training error can be made arbitrarily small with enough steps.
 \end{enumerate}
    
	\ifnum\solutions=1 {
	\begin{answer}
    \begin{enumerate}[label=(\roman*)]
        \item 
        Differentiate and set to zero
        
        \begin{equation*}
        \begin{align*}
            \frac{d}{d\tilde{w}}Z_t &= 0 = -(1 - \varepsilon) \exp(-\tilde{w}_t) + \varepsilon_t \exp(\tilde{w}_t)
        \end{align*}
        \end{equation*}

        Then,
        $$
            \tilde{w}_t^* = \frac{1}{2} \log(\frac{1-\varepsilon_t}{\varepsilon_t})
        $$
        
        So,
        $$
            Z_{t}^{\text{opt}} = (1 - \varepsilon) \sqrt{\frac{\varepsilon}{1 - \varepsilon}} + \varepsilon_t \sqrt{\frac{1 - \varepsilon}{\varepsilon}} = \boxed{2\sqrt{\varepsilon_t(1-\varepsilon_t)}}
        $$

        \item 
        \[
        Z_t^{\mathrm{opt}}
        = 2\sqrt{\varepsilon_t(1-\varepsilon_t)}
        = 2\sqrt{\left(\tfrac{1}{2}-\gamma_t\right)\left(\tfrac{1}{2}+\gamma_t\right)}
        = 2\sqrt{\tfrac{1}{4}-\gamma_t^2}
        = \sqrt{1-4\gamma_t^2}.
        \]
        Taking logarithms and using the inequality $\log(1-x) \le -x$ for $0 \le x < 1$:
        \[
        \log Z_t^{\mathrm{opt}} = \tfrac{1}{2}\log(1 - 4\gamma_t^2)
        \le \tfrac{1}{2}(-4\gamma_t^2)
        = -2\gamma_t^2.
        \]
        Exponentiating both sides gives
        \[
        \boxed{Z_t^{\mathrm{opt}} \le e^{-2\gamma_t^2}.}
        \]

        \item 
        If each weak classifier is better than random, i.e.\ $\gamma_t \ge \gamma > 0$ for all $t$, then
        \[
        \varepsilon_{\mathrm{training}}
        \;\le\;
        \prod_{t=1}^T Z_t
        \;\le\;
        \prod_{t=1}^T e^{-2\gamma_t^2}
        \;=\;
        e^{-2\sum_{t=1}^T \gamma_t^2}
        \;\le\;
        e^{-2T\gamma^2}.
        \]
        Thus
        \[
        \boxed{\varepsilon_{\mathrm{training}} \le e^{-2T\gamma^2},}
        \]

        \end{enumerate}
\end{answer}
        } \fi
        
    \item \subquestionpoints{4} 
Consider a training set X. In bootstrap sampling, each time we draw a random sample $Z$ of size N from the training data and obtain ${Z_1, Z_2, ..., Z_B}$ after $B$ times, i.e. we generate B different bootstrapped training data sets. If we apply bagging to regression trees, each time a tree $T_i (i = 1,2,...,B)$ is grown based on the bootstrapped data $Z_i$, and we average all the predictions to get:
\begin{align*}
    \hat{T(x)} =  \frac{1}{B}\sum_{i=1}^{B} T_i(x)
\end{align*}
Now, if $T_1, T_2,..., T_B$ is independent from each other, but each has the same variance $\sigma^2$, the variance of the average $\hat{T}$ is $\sigma^2/B$. However, in practice, the bagged trees could be similar to each other, resulting in correlated predictions. Assume $T_1, T_2,..., T_B$ still share the same variance $\sigma^2$, but have a positive pair-wise correlation $\rho$. We define the correlation between two random variables as:\\
\begin{align*}
    Corr(X,Y)=\dfrac{Cov(X,Y)}{\sqrt{Var(X)}\sqrt{Var(Y)}}
\end{align*}
Thus, we have $\rho = Corr(T_i(x), T_j(x)), i \neq j$.

Show that in this case, the variance of the average is given by:
\begin{align*}
    Var(\frac{1}{B}\sum_{i=1}^{B} T_i(x)) = \rho \sigma^2 + \frac{1-\rho}{B} \sigma^2
\end{align*}
    
	\ifnum\solutions=1 {
	\begin{answer}

The variance of the sum of random variables and the scaling property are:
\[
\operatorname{Var}\!\left(\sum_i X_i\right) 
= \sum_i \operatorname{Var}(X_i) + 2\sum_{i<j} \operatorname{Cov}(X_i, X_j),
\qquad
\operatorname{Var}(cX) = c^2 \operatorname{Var}(X).
\]

Covariance is defined as:
\[
\operatorname{Cov}(X, Y) = \operatorname{Corr}(X, Y)\,\sqrt{\operatorname{Var}(X)\operatorname{Var}(Y)}.
\]

Since all trees \(T_1, T_2, \dots, T_B\) share the same variance \(\sigma^2\) and pairwise correlation \(\rho\),
\[
\operatorname{Cov}(T_i, T_j) = \rho\,\sigma^2 \quad \text{for } i \neq j.
\]

Then, the variance of the average prediction is:
\begin{align*}
\operatorname{Var}\!\left(\frac{1}{B}\sum_{i=1}^B T_i\right) 
&= \frac{1}{B^2} \left( \sum_{i=1}^B \operatorname{Var}(T_i) + 2\sum_{i<j} \operatorname{Cov}(T_i, T_j) \right) \\
&= \frac{1}{B^2} \left( B\sigma^2 + 2 \cdot \frac{B(B-1)}{2}\rho\sigma^2 \right) \\
&= \frac{1}{B^2} \left( B\sigma^2 + B(B-1)\rho\sigma^2 \right) \\
&= \frac{\sigma^2}{B}\big(1 + (B-1)\rho\big) \\
&= \rho\sigma^2 + \frac{1-\rho}{B}\sigma^2.
\end{align*}


\end{answer}
        } \fi
        
\end{enumerate}
