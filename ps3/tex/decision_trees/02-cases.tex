\item \subquestionpoints{5} List out the conditions on the cardinality and proportion of positive and negative samples on the children where Gini loss will stay the same after a split. Show why these do not violate the strong concavity of the Gini loss.  Briefly explain why these cases do not prevent a fully grown tree from achieving zero Gini loss. (\textbf{Hint}: Recall the definition of strict concavity).