\begin{answer}
    \begin{enumerate}[label=\roman*.]
        \item 
        The semi-supervised EM converged in fewer iterations ($\approx 23$) than unsupervised EM,
        since the labeled data guide the parameter estimates from the start.
        The unsupervised EM required more iterations to reach a stable log-likelihood ($\approx 100$).

        \item 
        The semi-supervised EM was more stable: cluster assignments remained consistent across random initializations (Fig. 4, 5, 6),
        while the unsupervised EM swapped labels or converged to different local optima (Fig. 1, 2, 3).

        \item 
        The semi-supervised EM achieved better clustering quality.
        It correctly identified the three compact Gaussian components (red, blue, green) and the single high-variance one (yellow),
        whereas the unsupervised EM sometimes merged small clusters or split the large one.
        Overall, labeled data made the assignments more accurate and interpretable.
    \end{enumerate}

\end{answer}
