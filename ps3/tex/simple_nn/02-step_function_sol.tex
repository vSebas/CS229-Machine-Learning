\begin{answer}

If each hidden unit uses a step activation:
$$
h_j(x) = \mathbbm{1}\{\, w^{[1]}_{1,j}x_1 + w^{[1]}_{2,j}x_2 + w^{[1]}_{0,j} > 0 \,\},
\qquad j = 1,2,3,
$$
each one represents a linear inequality that defines one side of the triangle.

By setting the hidden weights and biases so that these three lines match the triangle’s edges, with normal vectors pointing inward, points inside the triangle satisfy all three inequalities, producing 
$h_1=h_2=h_3=1$. Points outside the triangle violate at least one of them.

The output neuron can then implement a logical AND:
$$
o(x) = \mathbbm{1}\{\, w^{[2]}_1 h_1 + w^{[2]}_2 h_2 + w^{[2]}_3 h_3 + w^{[2]}_0 > 0 \,\},
$$
for example with $w^{[2]}_1 = w^{[2]}_2 = w^{[2]}_3 = 1$, and $w^{[2]}_0 = -2.5$.

Then, the network outputs $1$ if and only if all three hidden units are active. Hence, the network can achieve $100\%$ training accuracy.

\begin{figure}
    \centering
    \includegraphics[width=0.8\textwidth]{/home/saveasmtz/Documents/CS229/ps3/tex/figs/step_weights.pdf}
    \caption{Step Weights}
    \label{fig:step_weights}
\end{figure}


In \texttt{src/simple\_nn/simple\_nn.py} the step activations are:
\[
h_1=\mathbbm{1}\{-0.5 + x_1 \ge 0\},\quad
h_2=\mathbbm{1}\{-0.5 + x_2 \ge 0\},\quad
h_3=\mathbbm{1}\{3.75 - x_1 - x_2 \ge 0\}.
\]
Since the dataset labels the exterior as $1$ (mean$(y)\approx0.77$), the output implements
$\text{NOT-AND}$:
\[
o=\mathbbm{1}\{\,2.5 - h_1 - h_2 - h_3 > 0\,\}=1-\mathbbm{1}\{h_1=h_2=h_3=1\},
\]
yielding perfect accuracy. Fig. \ref{fig:step_weights} shows the plot with the perfect prediction.


\end{answer}
