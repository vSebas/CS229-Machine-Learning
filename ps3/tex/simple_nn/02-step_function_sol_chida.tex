\begin{answer}

If each hidden unit uses a step activation:
$$
h_j(x) = \mathbbm{1}\{\, w^{[1]}_{1,j}x_1 + w^{[1]}_{2,j}x_2 + w^{[1]}_{0,j} > 0 \,\},
\qquad j = 1,2,3,
$$
we can interpret each hidden unit as testing one of the three half-spaces defined by the edges of the triangle.

Choose the hidden unit weights and biases so that each line
$$
w^{[1]}_{1,j}x_1 + w^{[1]}_{2,j}x_2 + w^{[1]}_{0,j} = 0
$$
coincides with one side of the triangle, with the normal vector pointing \emph{inward}.
Then any point inside the triangle satisfies all three inequalities and gives
$$
h_1 = h_2 = h_3 = 1,
$$
while a point outside violates at least one, producing at least one $h_j=0$.

The output neuron can then implement a logical AND:
$$
o(x) = \mathbbm{1}\{\, w^{[2]}_1 h_1 + w^{[2]}_2 h_2 + w^{[2]}_3 h_3 + w^{[2]}_0 > 0 \,\},
$$
with, for example,
$$
w^{[2]}_1 = w^{[2]}_2 = w^{[2]}_3 = 1, \qquad w^{[2]}_0 = -2.5.
$$
Thus, $o(x)=1$ if and only if all $h_j=1$, i.e., if the point lies inside the triangle.
Hence, the network can achieve $100\%$ training accuracy.

\end{answer}
