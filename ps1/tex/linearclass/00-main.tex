\item \points{15} {\bf Linear Classification (logistic regression)}

In this problem, we cover the linear classification task using logistic regression. The goal is to find a linear decision boundary that
separates the data into two classes. We will consider two datasets, along with starter codes provided in the following
files:
\begin{center}
\begin{itemize} %[label=\roman*.]
	\item \url{src/linearclass/ds1_{train,valid}.csv}
	\item \url{src/linearclass/ds2_{train,valid}.csv}
        \item \url{src/linearclass/logreg.py}
\end{itemize}
\end{center}
Each file contains $\nexp$ examples, one example $(x^{(i)}, y^{(i)})$ per row.
In particular, the $i$-th row contains columns $x^{(i)}_1\in\Re$,
$x^{(i)}_2\in\Re$, and $y^{(i)}\in\{0, 1\}$.

Typically, a trained model is evaluated by its performance on the validation dataset. The validation dataset is a set of examples drawn from the same (or a similar) distribution as the training data. Intuitively, this is because we need the trained model to correctly predict the label for not only the training data, but also new samples from the same distribution.

\begin{enumerate}
	\input{linearclass/01-logreg}
        \ifnum\solutions=1 {
            \begin{answer}

Let $\eta=\theta^\top x$. For one example $(x,y)$ with $y\in\{0,1\}$,
the (negative) log-likelihood (logistic loss) is
\[
\ell(\theta) \;=\; -\Big( y\log g(\eta) + (1-y)\log(1-g(\eta)) \Big).
\]

Gradient (w. chain-rule).
Differentiate w.r.t.\ $\eta$ first and then w.r.t. $\theta$:
\[
\frac{\partial \ell}{\partial \eta}
= -\Big( y\,\frac{1}{g(\eta)}\,\frac{\partial g(\eta)}{\partial \eta}
       + (1-y)\,\frac{1}{1-g(\eta)}\,\frac{\partial (1-g(\eta))}{\partial \eta} \Big)
= -\Big( y\,\frac{1}{g(\eta)}\,g(\eta)' \;-\; (1-y)\,\frac{1}{1-g(\eta)}\,g(\eta)' \Big),
\]
since $\dfrac{\partial (1-g(\eta))}{\partial \eta} = -g(\eta)'$. Using $g(\eta)'=\dfrac{\partial g(\eta)}{\partial \eta} = g(\eta)(1-g(\eta))$,
\[
\frac{\partial \ell}{\partial \eta}
= -\,y\,\frac{g(\eta)(1-g(\eta))}{g(\eta)} \;+\; (1-y)\,\frac{g(\eta)(1-g(\eta))}{1-g(\eta)}
= -(y(1-g(\eta))) \;+\; (1-y)g(\eta) 
= g(\eta) - y.
\]
Therefore,
\[
\nabla_\theta \ell(\theta)
= \frac{\partial \ell}{\partial \eta}\,\frac{\partial \eta}{\partial \theta}
= (g(\eta)-y)\,x.
\]

Hessian
\[
\nabla_\theta^2 \ell(\theta)
= \nabla_\theta\!\big((g(\eta)-y)\,x\big)
= \big(\nabla_\theta g(\eta)\big)\,x^\top
= \Big(\frac{\partial g(\eta)}{\partial \eta}\,\frac{\partial \eta}{\partial \theta}\Big)\,x^\top
= g'(\eta)\,x\,x^\top
= g(\eta)(1-g(\eta))\,x\,x^\top.
\]

Verification
For any $z\in\mathbb{R}^d$,
\[
z^\top \nabla_\theta^2 \ell(\theta)\, z
= z^\top (g(\eta)(1-g(\eta))\,x\,x^\top) z
= g(\eta)(1-g(\eta))\, (z^\top x)^2 \;\ge\; 0,
\]
because $g(\eta)(1-g(\eta))\ge 0$ and $(z^\top x)^2\ge 0$. Hence $\nabla_\theta^2 \ell(\theta)\succeq 0$.

%\paragraph{Dataset form (for completeness).}
%For $n$ examples $\{(x^{(i)},y^{(i)})\}_{i=1}^n$,
%\[
%J(\theta) \;=\; \frac{1}{n}\sum_{i=1}^n \ell^{(i)}(\theta),
%\qquad
%\nabla_\theta^2 J(\theta) \;=\; \frac{1}{n}\sum_{i=1}^n g(\eta)_i(1-g(\eta)_i)\,x^{(i)}{x^{(i)}}^\top,
%\]
%with $g(\eta)_i=g(\theta^\top x^{(i)})$. Thus for any $z$,
%\[
%z^\top \big(\nabla_\theta^2 J(\theta)\big) z
%= \frac{1}{n}\sum_{i=1}^n g(\eta)_i(1-g(\eta)_i)\,(z^\top x^{(i)})^2 \;\ge\; 0,
%\]
%so $\nabla_\theta^2 J(\theta)\succeq 0$.
% (Vectorized: with X having rows x^{(i)} and S=diag(h_i(1-h_i)), \; \nabla^2 J = \tfrac{1}{n} X^\top S X \succeq 0.)


\end{answer}

        } \fi

	\input{linearclass/02-solve-logreg}
        \ifnum\solutions=1 {
            \begin{answer}

Figure \ref{fig:logreg1} shows results for the first dataset, and Figure \ref{fig:logreg2}, for the second.
\begin{figure}
    \centering
    \includegraphics[width=0.75\linewidth]{ps1/tex/img/logreg_pred_1.png}
    \caption{LogReg Prediction 1}
    \label{fig:logreg1}
\end{figure}
\begin{figure}
    \centering
    \includegraphics[width=0.75\linewidth]{ps1/tex/img/logreg_pred_2.png}
    \caption{LogReg Prediction 2}
    \label{fig:logreg2}
\end{figure}
\end{answer}

        } \fi

\end{enumerate}
